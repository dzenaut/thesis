\documentclass{article}
    % General document formatting
    \usepackage[margin=0.7in]{geometry}
    \usepackage[parfill]{parskip}
    \usepackage[utf8]{inputenc}

    % Images
    \usepackage{graphicx}

    % Colors
    \usepackage{xcolor}
    
    % Related to math
    \usepackage{amsmath,amssymb,amsfonts,amsthm,bm}

\begin{document}

\subsection*{Dynamic Simulation}
\subsubsection*{TODOS}
\begin{itemize}
    \item What is dynamic simulation?
    \item What is the standard approach?
    \item What are the challenges?
    \item Enter constraints!
\end{itemize}

Simulating effects such as incompressibility, inextensibility and joints between atriculated rigid bodies 
can be achieved in elasticity-based simulations by using high stiffness values. High stiffness values lead to 
large forces which in turn cause numerical issues in the solver. 

We demonstrate these issues based on the example of maintaining a desired distance between two points using a stiff 
spring \cite{tournier2015}. Let $\bm{x_1, x_2}$ be the positions, $\bm{v_1, v_2}$ the velocities and $\bm{a_1, a_2}$
be the accelerations of the two particles. Let $\overline{l}$ be the rest length and $l = \lVert \bm{x_1} - \bm{x_2} \rVert$ 
be the current length of the spring with stiffness $k$. It can be shown that the force that the spring applies at each particle
is equal to $\bm{f_1} = -\bm{f_2} = \lambda\bm{u}$, where $\bm{u} = (\bm{x_1} - \bm{x_2}) / l$
and $\lambda = -\frac{\delta V}{\delta l} = k(\overline{l} - l)$. 

Once the forces, accelerations, velocities and positions are combined into a single vectors $\bm{f}, \bm{a}, \bm{v}, \bm{x}$, 
respectively, the motions of the system can be modeled via Newton's Ordinary Differential Equation (ODE) $\bm{f} = \bm{Ma}$,
where $\bm{M}$ is a $n_d \times n_d$ diagonal matrix and $n_d$ is the total number of independent degrees of freedom for the 
particles.

This system can be integrated via the symplectic Euler method as follows:

\begin{align*}
    \bm{v_{n+1}} &= \bm{v_n} + h\bm{a_n} \\
    \bm{x_{n+1}} &= \bm{x_n} + h\bm{v_{n+1}}
\end{align*}

As the stiffness $k$ of the spring increases, so does the magnitude of the acceleration $\bm{a}$. Consequently, the integration
diverges unless the timestep is prohibitively small. The stability issues are often addressed by switching to an implicit 
integration scheme, such as the backward Euler method \cite{baraff1998}. Replacing current accelerations with future accelerations
requires the solution of the following linear system of equations (LSE):

\[
    (\bm{M} - h^2\bm{K})\bm{\Delta v} = \bm{p} + h\bm{f}
\]

where $\bm{\Delta v} = \bm{v_{n+1}} - \bm{v_{n}}$, $\bm{p} = \bm{Mv}$ is the momentum, and $\bm{K} = \frac{\delta \bm{f}}{\delta \bm{x}}$ 
is the stiffness matrix. \textcolor{red}{Check whether using $\Delta v$ is correct here. Check \cite{baraff1998}}. Note that $\bm{K}$
is typically non-singular since elastic forces are invariant under rigid body transforms. When using large stiffness $k$ for springs, 
the entries of $\bm{K}$ are large (due to large restorative forces for stiff springs) and dominate the entries of the system matrix 
$\bm{H} = \bm{M} - h^2\bm{K}$. In these cases, $\bm{H}$ will be almost non-singular as well, leading to numerical issues and poor 
convergence for many solvers.

\subsection*{Constraint-based Dynamics}

\subsubsection*{Soft Constraints}



% Bibliography ------------------------------------------------------------------------

\bibliographystyle{plain}
\bibliography{bibliography/references}

\end{document}
