\chapter{Foundations}\label{ch:foundations}

In most approaches for the simulation of physical systems, the motion of the system is assumed to be in accordance with Newton's laws of
motion. Due to Newton's second law 

\begin{equation}\label{newtons-laws-2}
    \vecm{f} = m\vecm{a},
\end{equation}

\noindent it is possible to relate the force $\vecm{f}$ acting on a particle and the resulting acceleration $\vecm{a}$ via the particle
mass $m$. The motion of a particle system can then be described in terms of a system of ordinary differential equations (ODEs). This
system of ODEs is commonly referred to as the equations of motion. The equations of motion are integrated over time to arrive 
at the configuration of the system at the next time step. For general forces, finding an analytical solution to the equations of motion
is impossible and numerical integration schemes must be employed instead. In particular, both XPBD \textcolor{red}{(ref somewhere)} 
and PD \textcolor{red}{(ref somewhere)} are based on a numerical integration technique called implicit Euler integration 
\cite{macklin2016, bouaziz2014}. The equations of motion due to Newton's second law are reviewed in Section \ref{s:equations-of-motion}. Common 
approaches for numerical integration are introduced in Section \ref{ss:numerical-integration}. An alternative formulation of implicit Euler integration 
as an unconstrained optimization problem called the variational form of implicit Euler integration is presented in Section \ref{ss:variational-implicit-euler}. 
Additional considerations for the numerical integration of the equations of motion in real-time settings are covered in 
Section \ref{ss:numerical-integration-rt}. Simulating deformable bodies via numerical integration requires the computation of elastic forces from 
deformed configurations. We describe how to achieve this in Section \ref{ss:deformable-bodies}. The exact elastic forces do not only depend on 
the deformation of the body, but also on its material. A brief introduction to relevant material models is provided in Section \ref{ss:material-models}.
Lastly, it is often desirable to exert direct control over the positions of individual particles by imposing constraints on the simulation. Constrained 
dynamical simulation is covered in Section \ref{s:dynamic-simulation}.

\section{The Equations of Motion}\label{s:equations-of-motion}

The motion of a spatially discretized system with $m$ particles evolving in time according to Newton's laws can be modeled via 
the equations of motion

\begin{align}
    \begin{split}\label{eq:equations-of-motion}
        \dot{\vecm{q}}(t) &= \vecm{v}(t) \\
        \dot{\vecm{v}}(t) &= \matm{M}^{-1}\vecm{f}(\vecm{q}(t), \vecm{v}(t)),
    \end{split}
\end{align}

\noindent where $\vecm{q}(t)$, $\vecm{v}(t)$, $\vecm{f}(\vecm{q}(t), \vecm{v}(t))$ are the particle positions, particle velocities and 
forces acting on each particle at time $t$, respectively, and $\matm{M}$ is a diagonal matrix with the particle masses as diagonal entries. 
From now on, we only consider forces that are independent of velocities and write $\vecm{f}(\vecm{q}(t))$ instead of 
$\vecm{f}(\vecm{q}(t), \vecm{v}(t))$. Depending on the context, it is
$\vecm{q}(t), \vecm{v}(t), \vecm{f}(\vecm{q}(t)) \in \mathbb{R}^{m \times 3}$ and $\matm{M} \in \mathbb{R}^{m \times m}$ or 
$\vecm{q}(t), \vecm{v}(t), \vecm{f}(\vecm{q}(t)) \in \mathbb{R}^{3m}$ and $\matm{M} 
\in \mathbb{R}^{3m \times 3m}$. We define $\dot{\vecm{q}}(t) \coloneqq \frac{d\vecm{q}}{dt}(t)$ and $\dot{\vecm{v}}(t) \coloneqq 
\frac{d\vecm{v}}{dt}(t)$. When the time step $t$ is obvious from the context, we write $\vecm{q}$ and $\dot{\vecm{q}}$ instead of 
$\vecm{q}(t)$ and $\dot{\vecm{q}}(t)$ for time-dependent quantities for the sake of brevity.

The positions $\vecm{q}(t)$ and velocities $\vecm{v}(t)$ at time $t$ can be determined by solving the equations of motion given some initial conditions
$\vecm{q}(0) = \vecm{q}_0$ and $\vecm{v}(0) = \vecm{v}_0$. Here, $\vecm{q}_0$ and $\vecm{v}_0$ are the initial positions and velocitities, respectively. 
A function $\vecm{s}$ is a solution of the equations of motion if 

\[
    \vecm{s}(0) = 
    \begin{pmatrix}
        \vecm{q}(0)\\
        \vecm{v}(0)
    \end{pmatrix} 
    \text{ and } 
    \dot{\vecm{s}} = 
    \begin{pmatrix}
        \dot{\vecm{q}}\\
        \dot{\vecm{v}}
    \end{pmatrix}.
\]

\noindent Given an analytical solution $\vecm{s}$, the positions and velocities of the simulated system at time $t$ can be determined by computing 
$\vecm{s}(t)$. For general nonlinear forces, analytical solutions of the equations of motion are usually not available. Thus, the equations of motion 
need to be solved numerically.

\section{Numerical Integration of The Equations of Motion}\label{ss:numerical-integration}
Numerical integration schemes for the equations of motion aim at approximating the true positions $\vecm{q}(t_n)$ and velocities $\vecm{v}(t_n)$ 
at discrete points in time $t_n, n \in \mathbb{N}$ without computing the analytical solution $\vecm{s}$ directly. In the context of this 
thesis, $t_n = nh$ for some time step $h \in \mathbb{R}$. Note that the time step is constant. The approximations of $\vecm{q}(t_n)$ 
and $\vecm{v}(t_n)$ produced by the integration scheme are referred to as $\vecm{q}_n$ and $\vecm{v}_n$, respectively. In the following, we 
introduce explit, symplectic and implicit Euler integration in Section \ref{ss:explicit-euler}, \ref{ss:symplectic-euler} and 
\ref{ss:implicit-euler}, respectively.

\subsection{Explicit Euler Integration}\label{ss:explicit-euler}
When applied to Equation \ref{eq:equations-of-motion}, the next position and velocity estimates $\vecm{q}_{n+1}$ and $\vecm{v}_{n+1}$ are derived 
from the current estimates $\vecm{q}_n$ and $\vecm{v}_n$ via the update formula

\begin{align*}
    \vecm{q}_{n+1} &= \vecm{q}_n + h\vecm{v}_n\\
    \vecm{v}_{n+1} &= \vecm{v}_n + h\matm{M}^{-1}\vecm{f}(\vecm{q}_n).
\end{align*}

\noindent The idea is to simplify the integration of the functions $\dot{\vecm{q}}, \dot{\vecm{v}}$ over the time step $h$ by using constant 
approximations. Then, time integration is as simple as multiplying the constant function values 
with the time step $h$. In the explicit Euler method, we approximate $\dot{\vecm{q}}, \dot{\vecm{v}}$ by their current estimates $\vecm{q}_n , 
\vecm{v}_n$.

One main criterion for evaluating numerical integration schemes is the global trunction error \cite{chapra2005}, which is the error between 
$(\vecm{q}^T_n, \vecm{v}^T_n)^T$ and $(\vecm{q}(t_n)^T, \vecm{v}(t_n)^T)^T$ incurred during $n$ applications of the update formula. The global 
trunction error is the sum 
of the local trunction error (the error that results in the application of the method over a single time step $h$) and the propagated truncation 
error (the error resulting from the approximations produced during the previous steps). It can be shown that the global truncation error of 
explicit Euler integration over $n$ iterations is $\mathcal{O}(nh)$. Thus, explicit Euler integration is a first-order method. 

The main drawback of explicit Euler integration is that its estimates $\vecm{q}_n$ and $\vecm{v}_n$ can often diverge with growing $n$ unless 
prohibitely small time steps are used, even if the true functions $\vecm{q}$ and $\vecm{v}$ are bounded \cite{chapra2005}. Such integration schemes 
are called unstable. In the context of phyiscal simulations, it is also desirable that numerical integration schemes preserve important physical 
invariants of the equations of motion such as the conservation of momentum and energy. However, in the absence of artificial damping, explicit
Euler integration amplifies the energy of the simulated system. For example, it fails unconditionally on the undampened harmonic oscillator 
\cite{servin2006}. Using the explicit Euler method with larger time steps often manifests itself in exploding
simulations. These issues are particularly pronounced during the simulation of stiff systems, i.e. systems with large accelerations.

\subsection{Symplectic Euler Integration}\label{ss:symplectic-euler}
A variation of the explicit Euler method applied to the equations of motion, called the symplectic Euler method, arises when the new 
velocities $\vecm{v}_{n+1}$ instead of the old velocities $\vecm{v}_n$ are used in the position update \cite{stern2006}. This leads to 
the update formula

\begin{align}
    \begin{split}\label{eq:symplectic-euler}
        \vecm{q}_{n+1} &= \vecm{q}_n + h\vecm{v}_{n+1}\\
        \vecm{v}_{n+1} &= \vecm{v}_n + h\matm{M}^{-1}\vecm{f}(\vecm{q}_n).
    \end{split}
\end{align}

\noindent The symplectic Euler method has the same computational cost as the explicit Euler method. Just like explicit Euler integration, it 
is a first-order method. However, while it still does not conserve the system's energy exactly, it conserves a quadratic form that is close 
\cite{servin2006}. In contrast to the explicit Euler method, the symplectic Euler method preserves the amplitude of a swinging pendulum with 
appropriate time steps \cite{stern2006}. Its main drawback is that it also becomes unstable for stiff simulations unless the time step is 
kept prohibitively small \cite{servin2006}. 

\subsection{Implicit Euler Integration}\label{ss:implicit-euler}
Another popular integration scheme for tackling the equations of motion is implicit Euler integration, defined by the update formula

\begin{align}
    \begin{split}\label{eq:implicit-euler}
        \vecm{q}_{n+1} &= \vecm{q}_n + h\vecm{v}_{n+1}\\
        \vecm{v}_{n+1} &= \vecm{v}_n + h\matm{M}^{-1}\vecm{f}(\vecm{q}_{n+1}).
    \end{split}
\end{align}

\noindent Note how $\vecm{q}_{n+1}$ and $\vecm{v}_{n+1}$ appear on both sides of the equations. Consequently, performing implicit 
Euler integration includes solving a system of algebraic equations. For general nonlinear forces, the entire system of equations is nonlinear.
Thus, implicit Euler integration is more computationally expensive than both explicit and symplectic Euler integration. 
Despite the added complexity, implicit Euler integration is still only first-order accurate. However, implicit Euler integration 
remains a popular choice for physical simulation since it can be shown to be unconditionally stable \cite{chapra2005}. This allows 
dramatically increasing the size of the time steps. The additional cost of a single implicit Euler 
step compared to the previously mentioned integration schemes is offset by the fact that fewer steps are required to drive the 
simulation forward the same amount of time in a stable manner. It is worth poiting out that the stability guarantees of implicit Euler 
integration require the computation of the exact solution of the NLSE in Equation \ref{eq:implicit-euler}. Solving the NLSE commonly involves
solving derived LSEs whose system matrices are often ill-conditioned if the equations of motion are stiff (see Sec.\ \ref{ss:penalty-forces}). 
Thus, implicit Euler integration is still prone to numerical instabilities in practice. Finally, implicit Euler integration is known 
to exhibit numerical 
damping \cite{stern2006}. This manifests itself in a loss of energy in the simulated system. For example, the amplitude of a swinging 
pendulum decreases if integrated using the implicit Euler method. It is important to note that the extent of the observed numerical damping
depends on the time step $h$, making it difficult to control.

This brief discussion of numerical integration schemes shows that each of the introduced methods comes with their own sets of advantages and
drawbacks. In particular, each of the methods trade off stability in favor of the preservation of physical invariants such as the conservation
of energy or vice versa. Additionally, numerical integration of stiff system poses challenges for each of the methods introduced here. While
explicit and symplectic Euler integration are analytically unstable without prohibitively small time steps or additional
precautions, implicit Euler integration is prone to numerical instabilities for stiff systems. More sophisticated methods for physical 
simulation (see Sec.\ \ref{s:dynamic-simulation}) build on top of the integration schemes introduced here with the goal of alleviating their 
drawbacks without sacrificing their more advantageous properties. 

\section{Variational Form of Implicit Euler Integration}\label{ss:variational-implicit-euler}

Instead of solving the system of equations in Equation \ref{eq:implicit-euler} directly, it is possible to approach implicit
Euler integration via an equivalent unconstrained minimization problem over the particle positions at the next time step. This formulation
is called the variational form of implicit Euler integration \cite{bouaziz2014}. We summarize the derivation of the optimization problem
according to Bouaziz et al.\ \cite{bouaziz2014} below.

By rewriting the first line of Equation \ref{eq:implicit-euler} as

\[
    \vecm{v}_{n+1} = \frac{1}{h}(\vecm{q}_{n+1} - \vecm{q}_n)
\]

\noindent and substituting into the velocity update of Equation \ref{eq:implicit-euler} the following equation can be derived

\begin{equation}\label{eq:implicit-positional}
    \matm{M}(\vecm{q}_{n+1} - \vecm{q}_n - h\vecm{v}_n) = h^2(\vecm{f}(\vecm{q}_{n+1})).
\end{equation}

\noindent We separate forces $\vecm{f}(\vecm{q})$ into internal forces $\vecm{f}_{\text{int}}(\vecm{q}) = \sum_{i \in \mathcal{I}_{\text{int}}} 
\vecm{f}^i_{\text{int}}
(\vecm{q})$ and external forces $\vecm{f}_{\text{ext}}(\vecm{q}) = \sum_{i \in \mathcal{I}_{\text{ext}}} \vecm{f}^i_{\text{ext}}(\vecm{q})$ 
with index sets $\mathcal{I}_{\text{int}}$ and $\mathcal{I}_{\text{ext}}$. We consider all external forces to be constant. Internal forces 
are conservative and defined in terms of scalar potential energy functions 
$\psi_j$ via $\vecm{f}^j_{\text{int}}(\vecm{q}) = -\nabla \psi_j(\vecm{q})$. Together, we have $\vecm{f} (\vecm{q}) 
= \vecm{f}_{\text{int}} (\vecm{q}) + \vecm{f}_{\text{ext}} = -\sum_j \nabla \psi_j(\vecm{q}) + \vecm{f}_{\text{ext}}$. Plugging into 
Equation \ref{eq:implicit-positional}, it is

\begin{equation}\label{eq:implicit-positional-detailed}
    \matm{M}(\vecm{q}_{n+1} - \vecm{q}_n - h\vecm{v}_n) = h^2\left(\vecm{f_\text{ext}} - \sum_j \nabla \psi_j(\vecm{q}_{n+1})\right).
\end{equation}

\noindent By computing first-order optimality conditions, it is easily verified that the system of equations above is equivalent to the optimization 
problem

\begin{equation}\label{eq:variational-implicit}
    \min_{\vecm{q}_{n+1}} \frac{1}{2h^2} \norm{\matm{M}^{\frac{1}{2}}(\vecm{q}_{n+1} - \tilde{\vecm{q}})}^2_F + \sum_j \psi_j(\vecm{q}_{n+1}).
\end{equation}

\noindent where $\tilde{\vecm{q}} = \vecm{q}_n + h\vecm{v}_n + h^2\matm{M}^{-1}\vecm{f}_{\text{ext}}$.  
We refer to the first and second term of the objective function as the inertial term and the constraint energy, respectively \cite{bouaziz2014}. 
Thus, the minimization problem requires that the solution minimizes the elastic deformation as best as possible while ensuring that the solution is 
close to following its momentum plus external forces. The weighting between the inertial term and the constraint energy depends on the 
particle masses $\matm{M}$, the time step $h$ and the material stiffness of the elastic potentials $\psi_j$. Larger masses and smaller time steps 
increase the weighting of the inertial term, whereas larger material stiffness increases the weighting of the constraint energy. 
The solution preserves linear and angular momentum as long as the elastic potentials are rigid motion invariant \cite{bouaziz2014}.

Approaching implicit Euler integration in its variational form can often be advantageous. That is because the objective function of 
Equation \ref{eq:variational-implicit} presents a natural merit function that can be employed to improve the step size along the search direction 
that arises in common unconstrained optimization algorithms based on Newton's method \cite{nocedal2006}. As an example, the objective 
function is used in many step length selection algorithms to ensure that step sizes satisfy the Wolfe conditions. In fact, many algorithms for 
solving nonlinear systems of equations construct approximate merit functions if the objective function from an equivalent optimization problem 
is not available \cite{nocedal2006}.


\section{Considerations for Real-Time Simulations}\label{ss:numerical-integration-rt}
As discussed in Section \ref{ss:numerical-integration}, both the explicit and symplectic Euler integrators exhibit instabilities when simulating
stiff systems without keeping the time step $h$ infeasibly small or adding artificial damping terms. On the other hand, implicit Euler integration
is unconditionally stable, even though numerical instabilities might still occur. Since unstable simulations cannot simply be restarted and 
recovered from in the real-time setting, stability is of utmost importance. For that reason, physical simulations for real-time applications
are commonly based on implicit Euler integration. However, this comes at the cost of artificial numerical damping and significant computational
overhead. According to Tournier et al.\ \cite{tournier2015}, it is common that the time budget allotted for integration over one time step is 
barely enough to perform a single linearization of the NLSE in Equation \ref{eq:implicit-euler}. Thus, additional measures are required to make
implicit Euler integration viable for real-time applications. 

There are two main strategies for speeding up the integration step. The first one is to simply approximate the solution of the NLSE in
Equation \ref{eq:implicit-euler}. In the simplest case, this consists of solving a linearized version of Equation \ref{eq:implicit-euler} and updating 
the new positions $\vecm{q}_{n+1}$ and velocities $\vecm{v}_{n+1}$ with the resulting values without iterating any further \cite{baraff1998}. 
Applying first-order Taylor expansion to the forces yields the following LSE

\begin{equation}\label{eq:linear-implicit-lse}
    \begin{split}
    \Delta \vecm{q} &= h \vecm{v}_{n+1} \\
    \Delta \vecm{v} &= h \matm{M}^{-1}(\vecm{f}(\vecm{q}_n) + \nabla_q \vecm{f}(\vecm{q}_n) \Delta \vecm{q}),
    \end{split}
\end{equation}

\noindent where $\Delta \vecm{q} = \vecm{q}_{n+1} - \vecm{q}_n$ and $\Delta \vecm{v} = \vecm{v}_{n+1} - \vecm{v}_n$. By rearranging terms, it is 
easy to show that $\vecm{v}_{n+1}$ can be determined by solving

\begin{equation}\label{eq:linear-implicit-stepper}
    (\matm{M} - h^2\matm{K})\vecm{v}_{n+1} = \matm{M}\vecm{v}_n + h\vecm{f}(\vecm{q}_n),
\end{equation}

\noindent where $\matm{K} = \frac{d\vecm{f}}{d\vecm{q}}$. The new positions can be computed via $\vecm{q}_{n+1} = h \vecm{v}_{n+1}$. 
Borrowing the terminology from \cite{servin2006}, we refer to this approach as the linear implicit Euler stepper. Note that iteratively solving
the linearized system of equations in Equation \ref{eq:linear-implicit-lse} and updating $\vecm{q}_{n+1}, \vecm{v}_{n+1}$ as described converges to 
the true solution of the NLSE for implicit Euler integration (see Eq.\ \ref{eq:implicit-euler}). The strategy of approximating the solution of 
Equation \ref{eq:implicit-euler} is also at the heart of XPBD (see Sec.\ \ref{ss:xpbd-constraint-projection}), even though the manner in which the 
approximation is achieved differs quite significantly from the linear implicit Euler stepper.

The second strategy is to compute the exact solution of Equation \ref{eq:implicit-euler}, but to restrict to a subset of forces whose strucuture 
allows speeding up the process of solving the NLSE. As an example, note that Equation \ref{eq:implicit-euler} is linear if the featured forces 
are linear. In this case, $\vecm{q}_{n+1}, \vecm{v}_{n+1}$ can be determined in a single linear solve. A similar approach, but with a less 
restrictive subset of forces that allows modelling nonlinearities, is used in PD (see Sec.\ \ref{s:pd}). The second method is particularly appealing 
in settings where visual plausibility is prioritized over physical accuracy.

\section{Elastic Forces For Deformable Bodies}\label{ss:deformable-bodies}
To be able to perform physically accurate simulations of deformable bodies using the numerical integration schemes for the equations of motion 
introduced in Section \ref{ss:numerical-integration}, we need to be able to compute elastic forces $\vecm{f}(\vecm{q})$ from the discretized 
positions $\vecm{q}$ of a potentially deformed body. Note that the we assume that the elastic forces $\vecm{f}(\vecm{q})$ are determined 
entirely by the current positions $\vecm{q}$. In particular, this implies that the elastic forces are path-independent. Such materials are called 
hyperelastic. In the following exposition, we lean heavily on the SIGGRAPH course by Sifakis et al.\ \cite{sifakis2012}. We start by 
introducing relevant concepts from continuum mechanics and then describe how efficient implementation can be achieved via suitable 
discretization schemes.

First, we introduce the deformation function $\vecm{\phi} \colon \Omega \mapsto \mathbb{R}^3$, where $\Omega \subseteq \mathbb{R}^3$ is the 
volumetric domain occupied by the undeformed body. This domain $\Omega$ is called the reference configuration. The deformation 
function maps every point in the reference configuration 
to its respective deformed position $\vecm{q} = \vecm{\phi}(\vecm{Q})$. Note that positions in the reference configuration are denoted using capital 
letters while the deformed positions are referred to via small letters. As an example, the deformation function given by 

\[
    \vecm{\phi}(\vecm{Q}) = 2\matm{I}\vecm{Q}
\]

\noindent describes the deformed configuration that is achieved by stretching the reference domain by a factor of 2 in each direction. 

The next step is to associate the deformation function that describes the current deformed configuration with an energy. However, the degree to 
which different parts of the body are deformed may vary. Thus, we need to be able to relate deformations and strain energies on a local scale. 
This can be achieved by defining an energy density function $\Psi(\vecm{\phi}, \vecm{Q})$ which measures the strain energy per unit undeformed volume 
on an infitisemal domain $dV$ around $\vecm{Q}$. Since the energy density function only depends on the local behavior of $\vecm{\phi}$ around 
$\vecm{Q}$, it can be equivalently expressed in terms of the deformation gradient $\matm{F}(\vecm{Q}) \in \mathbb{R}^{3 \times 3}$ given by 

\begin{equation}\label{eq:deformation-gradient}
    \matm{F} = \matm{F}(Q) = \frac{d\vecm{\phi}}{d\vecm{Q}},
\end{equation}

\noindent yielding $\Psi(\vecm{\phi}, \vecm{Q}) = \Psi(\matm{F}(\vecm{Q}))$. The choice of the energy density depends on the simulated material and 
is treated as a black box for now. Relevant material models are introduced in Section \ref{ss:material-models}. Now, the elastic energy stored in the 
entirety of the deformed body can be computed by integrating the energy density function over the reference configuration $\Omega$ like so

\begin{equation}\label{eq:elastic-energy}
    E(\vecm{\phi}) = \int_{\Omega}\Psi(\matm{F}) d\vecm{Q}.
\end{equation}

For general nonlinear materials, computing the integral in Equation \ref{eq:elastic-energy} is infeasible. Thus, simplifying assumptions are required 
in order to enable efficient simulation of the physics of deformable bodies. Firstly, we describe the reference configuration $\Omega$ as a tetrahedral 
mesh. Let $\vecm{Q} = (\vecm{Q}_1, \ldots, \vecm{Q}_n)^T$ be the vertex positions of the undeformed mesh. Then, the current 
state of the model can be expressed via the vertex positions in the deformed configuration $\vecm{q} = (\vecm{\phi}(\vecm{Q}_1), \ldots, 
\vecm{\phi}(\vecm{Q}_n))^T = (\vecm{q_1}, \ldots, \vecm{q}_n)^T$. Instead of using the true, complex deformation function $\vecm{\phi}$, we construct 
an approximation $\hat{\vecm{\phi}}$ from $\vecm{Q}$ and $\vecm{q}$ using barycentric interpolation. The reconstructed deformation map $\hat{\vecm{\phi}}$
is a piecewise linear function over each tetrahedron given by 

\[
    \hat{\vecm{\phi}}(\vecm{Q}) = \matm{F}_i \vecm{Q} + \vecm{b}_i \text{ for all } \vecm{Q \in \mathcal{T}_i},
\]

\noindent where $\matm{F}_i \in \mathbb{R}^{3 \times 3}$ and $\vecm{b}_i$ are specific to each tetrahedron $\mathcal{T}_i$. Note that $\matm{F}_i$ 
can be interpreted as the constant deformation gradient for all positions in the $i$-th tetrahedron. Let $\vecm{Q}_1, \ldots, \vecm{Q}_4$ and 
$\vecm{q}_1, \ldots, \vecm{q}_4$ denote the positions of the vertices of tetrahedron $\mathcal{T}_i$ in the reference configuration and the deformed 
configuration, respectively. Sifakis \cite{sifakis2012} shows that $\matm{F}_i$ is given by

\begin{equation}\label{eq:deformation-gradient-tet}
    \matm{F}_i = \matm{D}_s\matm{D}^{-1}_m
\end{equation}

\noindent where 

\[
    \matm{D}_s = \begin{pmatrix}
        q^x_1 - q^x_4 & q^x_2 - q^x_4 & q^x_3 - q^x_4 \\
        q^y_1 - q^y_4 & q^y_2 - q^y_4 & q^y_3 - q^y_4 \\
        q^z_1 - q^z_4 & q^z_2 - q^z_4 & q^z_3 - q^z_4
    \end{pmatrix}
\]

\noindent is the deformed shape matrix and 

\[
    \matm{D}_m = \begin{pmatrix}
        Q^x_1 - Q^x_4 & Q^x_2 - Q^x_4 & Q^x_3 - Q^x_4 \\
        Q^y_1 - Q^y_4 & Q^y_2 - Q^y_4 & Q^y_3 - Q^y_4 \\
        Q^z_1 - Q^z_4 & Q^z_2 - Q^z_4 & Q^z_3 - Q^z_4
    \end{pmatrix}
\]

\noindent is called the reference shape matrix.

Now, the elastic energy of the deformed body can be computed as a sum over the tetrahedral elements of the mesh via

\[
    E(\vecm{q}) = \sum_e E_e(\vecm{q}) = \sum_e \int_{\Omega_e} \Psi(\hat{\matm{F}}(\vecm{Q}, \vecm{q})) d\vecm{Q},
\]

\noindent where $\hat{\matm{F}}(\vecm{Q}, \vecm{q})$ is the deformation gradient of the reconstructed deformation 
function $\hat{\vecm{\phi}}$. Since the deformation gradient $\hat{\matm{F}}$ is constant across each tetrahedral element, 
it is further

\[
    E(\vecm{q}) = \sum_e V_e \Psi(\matm{F}_e),
\]

\noindent where $V_e$ is the volume of the undeformed tetrahedral element $e$. In contrast to Equation \ref{eq:elastic-energy}, this can be computed efficiently. 
Finally, we are able to specify the elastic forces $\vecm{f}(\vecm{q})$ acting on the particles of the deformed body with positions $\vecm{q}$ by computing

\[
    \vecm{f}(\vecm{q}) = -\frac{dE(\vecm{\vecm{q}})}{d\vecm{q}}.
\]

\section{Material Models}\label{ss:material-models}
In Section \ref{ss:deformable-bodies}, we demonstrated how to compute the elastic forces acting on a particle system with deformed positions 
$\vecm{q}$, but treated the material-dependent energy density function $\Psi$ as a black box. In this section, we fill in the gaps and describe the material
models relevant for this thesis. Note that the material models covered here are isotropic, i.e. their energy density functions satisfy $\Psi(\matm{F}) = 
\Psi(\matm{R}\matm{F})$ for any rotation matrix $\matm{R}$ \cite{sifakis2012}. Informally, this means that the response to a deformation is 
independent of the orientation the deformation is applied in. Additionally, we focus on material models that generate elastic forces that are nonlinear 
in the deformation gradient. In the following, we introduce the strain material model \cite{bouaziz2014}, the classical Neohookean model and a simplified 
variation thereof \cite{smith2018}.

\subsection{Strain Material Model}\label{ss:strain-material}
The first material model, which we simply refer to as the strain material model \cite{bouaziz2014}, is given by

\begin{equation}\label{eq:strain-material-model}
    \Psi_{\text{strain}}(\vecm{F}) = \frac{k}{2} \norm{\matm{U}\matm{I}\matm{V}^T - \matm{U}\matm{\Sigma}\matm{V}^T}^2_F 
    = \frac{k}{2} \norm{\matm{I} - \matm{\Sigma}}^2_F,
\end{equation}

\noindent where $\matm{F} = \matm{U}\matm{\Sigma}\matm{V}^T$ is the singular value decomposition of the deformation gradient $\matm{F}$ and $k 
\in \mathbb{R}^+$ is a stiffness parameter. Note that $\Psi_{\text{strain}}(\vecm{F}) = 0$ iff $\matm{\Sigma} = \matm{I}$ or equivalently iff 
$\matm{F}$ is a rotational matrix. It is easy to verify that $\matm{F}$ is a rotation iff the deformation function $\vecm{\phi}$ is a rigid body 
transform. Thus, Equation \ref{eq:strain-material-model} can be interpreted as a quadratic penalty for the distance between the 
deformation function $\vecm{\phi}$ and its closest rigid-body transform as measured by the Frobenius norm of the difference of their deformation 
gradients. 

Since the PD solver cannot be applied to classical material models like the St.\ Venant-Kirchhoff and the Neohookean model, the strain 
material model is the main tool for achieving simulations of deformable bodies using PD (see Sec.\ \ref{s:pd}).
While simple, the strain material model comes with a couple of disadvantages. Firstly, the computation of the singular value decomposition of 
$\matm{F}$ is expensive. Secondly, the restorative elastic forces arising from the strain material model are proportional 
to the aformentioned distance between the deformation and its closest rigid-body transform. As such, the strain material model is too 
restrictive to model complex materials such as skin or flesh \cite{liu2017}. 

\subsection{Neohookean Material Model}\label{ss:neohookean-material}
A popular material model that is more suitable for simulating biological materials is the Neohookean material model and its variations 
\cite{smith2018}. Its popularity is reflected by the fact that it has been implemented with a variety of different solvers including 
PBD \cite{bender2014}, XPBD \cite{macklin2021}, ADMM \cite{overby2017} and a quasi-Newton solver based on PD \cite{liu2017}.
The classical version of the Neohookean material model is given below

\begin{equation}\label{eq:neohookean-material-model}
    \Psi_{\text{Neo}} = \frac{\mu}{2}(\text{tr}(\matm{F}^T\matm{F}) - 3) - \mu \text{log}(\text{det}(\matm{F})) 
    + \frac{\lambda}{2}\text{log}(\text{det}(\matm{F}))^2.
\end{equation}

\noindent Here, $\mu$ and $\lambda$ are the material-specific Lamé parameters. The first term is minimized if $\matm{F} = \matm{0}$, which 
happens if the deformed volume has collapsed to a single point. Note that $\text{det}(\matm{F})$ can be considered a measure for the 
infitisemal local change in volume caused by the deformation $\vecm{\phi}$. Then, the last two terms can be interpreted as volume-preserving, 
since $\text{log}(\text{det}(\matm{F}))$ approaches negative infinity as $\text{det}(\matm{F})$ approaches $0$. In particular, 
the volume-preserving terms strongly counteract the compression of the modelled material encouraged by the first term. This way, the 
classical Neohookean model achieves rest stability.

The main issue with the Neohookean material model is that the volume-preserv-ing terms can grow unbounded when $\text{det}(\matm{F})$ approaches 
$0$ and are infeasible for $\text{det}(\matm{F}) < 0$. Note that while configurations where $\text{det}(\matm{F}) < 0$ are theoretically 
impossible to achieve, they can often occur in practice as a result of geometric constraints, instability of time integration techniques or 
inadequate convergence of numerical solvers \cite{sifakis2012}. Thus, $\Psi_{\text{Neo}}$ is not inversion stable \cite{smith2018}.

\subsection{Simplified Neohookean Material Model}\label{ss:simplified-neohookean-material}
To address the issues with the classical Neohookean material model, Smith et al.\ propose a modified version of $\Psi_{\text{Neo}}$ where 
the volume-preserving terms are replaced by the bounded, well-defined and invertible term $\frac{\lambda}{2}(\text{det}(\matm{F}) - 1)^2$. 
This yields the following energy density

\[
    \Psi(\matm{F}) = \frac{\mu}{2}(\text{tr}(\matm{F}^T\matm{F}) - 3) + \frac{\lambda}{2}(\text{det}(\matm{F})-1)^2.
\]

\noindent While this ensures inversion stability, Smith et al.\ show that its Piola-Kirchhoff tensor 

\[
    \matm{P}(\matm{F}) = \frac{d\Psi(\matm{F})}{d\matm{F}}
\]

\noindent is non-zero at undeformed configurations, i.e.\ $\matm{P}(\matm{I}) \neq \matm{0}$. Consequently, $\Psi$ is not rest stable. In particular,
the effective rest state is achieved when elements are slightly deflated. The authors correct this by adjusting the volume-preserving term so that 
elements are inflated again, yielding the following energy density

\begin{equation}\label{eq:smith-neohookean-material-model}
    \Psi_{\text{Smith}}(\matm{F}) = \frac{\mu}{2}(\text{tr}(\matm{F}^T\matm{F}) - 3) + \frac{\lambda}{2}\left(\text{det}(\matm{F}) - 
    \left(1 + \frac{\mu}{\lambda}\right)\right)^2.
\end{equation}

\noindent We refer to this material model as the simplified Neohookean model.

Equipped with these energy density functions, we have all the tools available that are required to compute the elastic forces acting on the particles 
of a deformed body as described in Section \ref{ss:deformable-bodies}. Note that the material models covered here are by no means exhaustive. Again, the 
SIGGRAPH tutorial by Sifakis \cite{sifakis2012} is a good place to gain a better overview for the interested reader.

\section{Constrained Dynamic Simulation}\label{s:dynamic-simulation}
Visually pleasing and physically accurate simulations of dynamical systems require modeling a wide range of materials and 
interactions between participating bodies. Common requirements that prove challenging are the ability to simulate stiff materials
and to enforce hard constraints on the positions of the simulated bodies. Among others, hard constraints involve attaching particles
to a fixed point in space or to each other, implementing joints and resolving interpenetrations between bodies. Unfortunately, in their 
naive form, the numerical integration schemes from Section \ref{ss:numerical-integration} are unsuitable for application to simulations 
including these effects. The main difficulty lies in the fact that stiff materials and hard constraints introduce stiffness into 
the simulated system, causing stability issues and numerical problems. This is obvious for stiff materials, where even small 
deformations can cause large restorative forces. For hard constraints, note that to fit into the equations of motion in 
Equation \ref{eq:equations-of-motion}, corresponding forces that ensure that the constraints stay satisfied at all times need to be defined. 
It is possible to show that hard constraints are the physical limit of strong potential forces \cite{servin2006}, increasing the 
stiffness of the equations of motion. 

In this section, we focus on different strategies for modelling hard constraints in physical real-time simulations and discuss 
their properties. Heavily inspired by Tournier et al.\ \cite{tournier2015}, the approaches covered in this section are analyzed 
in the context of real-time adaptions of implicit Euler integration that only perform a single linear solve per time step 
(see Sec.\ \ref{ss:numerical-integration-rt}). While both XPBD (see Sec.\ \ref{ss:xpbd-constraint-projection}) and PD (see Sec.\ \ref{s:pd}) differ significantly 
from these methods, the challenges encountered in modelling hard constraints are largely transferrable. Handling hard constraints 
via stiff penalty forces as outlined above is covered in more detail in Section \ref{ss:penalty-forces}. Adding hard constraints to the equations
of motion without introducing stiff constraint forces via a Differential Algebraic Equation (DAE) is explored in Section \ref{ss:hard-constraints}.
Finally, adding regularization to the resulting DAE in a physcially meaningful manner yields a method called compliant constraints. 
Compliant constraints are introduced in Section \ref{ss:compliant-constraints}.

\textcolor{red}{Probably best to skip generality and simplicitiy here. Don't distract from hard constraints. Make sure this is 
mentioned in the discussion of XPBD, PD, etc.}

\subsection{Penalty Forces}\label{ss:penalty-forces}
As discussed in Section \ref{s:dynamic-simulation}, it is possible to show that hard constraints are the physical limit of strong potential 
forces \cite{servin2006}. Thus, the simplest approach to modelling hard constraints is to introduce strong forces that enforce 
that the constraints are satisfied at all times to the equations of motion and to perform integration via the linear implicit Euler
stepper from Section \ref{ss:numerical-integration-rt}. We demonstrate the challenges of this method using the example of 
a distance constraint.

Consider a distance constraint $C_{\text{dist}}$ that 
fixes the distance $l = \vert \vert \vecm{q}_i - \vecm{q}_j \vert \vert$ between two particles $\vecm{q}_i$ and $\vecm{q}_j$ to 
some distance $d \in \mathbb{R}^+$. If $C_{\text{dist}}(\vecm{q}) = l - d$, then the constraint is satisfied if 
$C_{\text{dist}}(\vecm{q}) = 0$. Next, we define the energy potential $\psi_\text{dist}(\vecm{q}) = \frac{k}{2}C_\text{dist}(\vecm{q})^2$,
where $k \in \mathbb{R}^+$ is some large stiffness value. Since the energy potential is zero if $C_\text{dist}$ is satisfied
and grows as the distance between $\vecm{q}_i$ and $\vecm{q}_j$ increases, the resulting force will drive the distance towards $d$.
Upon closer inspection, $\psi_\text{dist}$ is exactly the potential energy of a Hookean spring with stiffness $k$. The forces 
$\vecm{f}_i$ and $\vecm{f}_j$ applied to the particles $\vecm{q}_i$ and $\vecm{q}_j$ by the spring are given by 
$\vecm{f}_i = -\vecm{f}_j = \beta (\vecm{q}_i - \vecm{q}_j) / l$, where $\beta = -\partial\psi_\text{dist}/\partial l = 
kC_\text{dist}(\vecm{q})$. Since $\vert \vert \vecm{f}_i \vert \vert = \vert \vert \vecm{f}_j \vert \vert = \beta$,
the restorative spring forces have infinite magnitude in the limit of infinite stiffness $k$ if $C_\text{dist}(\vecm{q}) \neq 0$. 

Integration using the linear implicit Euler stepper is achieved by plugging the spring forces into 
Equation \ref{eq:linear-implicit-stepper} and solving for the new velocities $\vecm{v}_{n+1}$. For the sake of convenience, we restate
Equation \ref{eq:linear-implicit-stepper} below

\[
    (\matm{M} - h^2\matm{K})\vecm{v}_{n+1} = \matm{M}\vecm{v}_n + h\vecm{f}(\vecm{q}_n).
\]

\noindent Note that $\matm{K}$ is typically non-singular since elastic forces are invariant under rigid body transforms. When using 
large stiffness values $k$, the entries of $\matm{K}$ are large (due to large restorative forces) and dominate the entries of the 
system matrix 

\begin{equation}\label{eq:system-H}
    \matm{H} = \matm{M} - h^2\matm{K}.
\end{equation}

\noindent In these cases, $\matm{H}$ is almost non-singular as well, leading to numerical issues and poor convergence for many solvers. 

In the example above, the potential energy $\psi_\text{dist}$ is defined in terms of the constraint function $C_\text{dist}$ via 
$\psi_\text{spring}(\vecm{q}) = \frac{k}{2} C_\text{dist}(\vecm{q})^2$. However, it is possible to generalize this approach to other 
geometric constraints $C$ yielding energy potentials of the form

\begin{equation}\label{eq:penalty-forces}
    \psi(\vecm{q}) = \frac{k}{2} C(\vecm{q})^2.
\end{equation}

\noindent The resulting forces are called penalty forces.

By constructing different geometric displacement functions, various properties such as the bending angle between triangles and in-plane 
shearing of triangles can be controlled via the corresponding quadratic energy potentials \cite{baraff1998}. Geometric displacement 
functions with the desired effect are often intuitive and simple to define. However, as the corresponding energy potentials are not physically 
derived, choosing stiffness parameters that correspond to measurable physical properties of the simulated material and orchestrating 
multiple constraints becomes challenging \cite{servin2006, nealen2006}. Additionally, the generated penalty forces do not converge in the 
limit of infinite stiffess, leading to oscillations unless the time step is reduced significantly \cite{rubin1957}. Finally, the numerical
issues outlined for distance constraints generally apply to penalty forces derived from other geometric constraints as well.

\textcolor{blue}{Maybe explain the challenges with penalty forces a bit better! Also read \cite{terz1987, nealen2006, rubin1957}. 
I just skimmed over \cite{terz1987} for now, but want to make sure that I am citing this correctly. The term penalty forces is not used
in the paper, I am just following the trail from \cite{servin2006}. \cite{nealen2006} is a review that might be intersting to read.
\cite{rubin1957} would be really interesting to read for once, just to understand why strong penalty forces oscillate. Is this a
general problem with penalty forces, or is it an issue with the solver?}

\subsection{Hard Constraints}\label{ss:hard-constraints}
The problem of maintaining hard distance constraints between particles can be formulated as a Differential Algebraic Equation (DAE)
\cite{ascher1995, baraff1996}. In this framework, the equations of motion \textcolor{red}{(reference somewhere)} are handled together 
with algebraic equations that model the constraints on the positions of the system. Distance constraints are typically implemented using 
holonomic constraints 
of the form $\vecm{\phi(x)} = 0$. Note that the distance constraint $\vecm{\phi}(x)$ is formulated in terms of the particle positions, 
whereas the equations of motion work on velocities and accelerations. Consequently, the constraints need to be differentiated with 
respect to time once or twice so that they can be 
combined with the ODE in terms of velocties or accelerations, respectively. \textcolor{blue}{In XPBD, we go the other way! The ODE
is tanslated so that it is in terms of positions, so that it can be handled together with the constraints. Is there a reason nobody 
bothered to do this before?} \textcolor{red}{Fabian notes that the translation in terms of positions in XPBD is in fact just rewriting
the implicit Euler integration in terms of positions. Thus, we have baked the integration method into the formulation in XPBD. In the DAE
framework, there is more flexibility w.r.t.\ the integration scheme of choice}. Using $\matm{J} = \frac{\delta \vecm{\phi}}{\delta \vecm{x}}$, 
where $\matm{J}$ is a $n_c \times n_d$ matrix and $n_c$ is the number of scalar constraints, this leads to the following constraint 
formulations:

\begin{align*}
    \matm{J}\vecm{v} &= 0 \\
    \matm{J}\vecm{a} &= \vecm{c}(\vecm{v})
\end{align*}

for some $\vecm{c}(\vecm{v})$. \textcolor{red}{If you check \cite{ascher1995}, see that $c(v)$ also depends on the positions $q$. That should 
be indicated!} 
Additionally, constraint forces \textcolor{red}{(use internal forces, more general and will be used throughout the thesis)} are 
required to link the algebraic constraint equations with the ODE describing the motion of 
the system. It can be shown that the constraint forces $\vecm{f_c}$ applied to the particles have to be in the following form to 
avoid adding linear and angular momentum to the system \cite{baraff1996}:

\begin{equation}\label{eq:constraint-forces}
    \vecm{f_c} = \matm{J}^T \vecm{\lambda}
\end{equation}

where the $\lambda$ are the Lagrange multipliers of the constraints. With external forces $\vecm{f}_{\text{ext}}$, the DAE can now be 
expressed as follows 
\cite{ascher1995}:

\[
    \begin{pmatrix}
        \matm{M} & -\matm{J}^T \\
        \matm{J} & 0
    \end{pmatrix}
    \begin{pmatrix}
        \vecm{a} \\
        \vecm{\lambda}
    \end{pmatrix}
    =
    \begin{pmatrix}
        \vecm{f_e} \\
        \vecm{c}(\vecm{v})
    \end{pmatrix}
\]

Note that the lower block-row of the system drives towards accelerations that satisfy the constraints imposed by $\vecm{\phi(x)}$ (or, striclty 
speaking, the differentiations thereof) exactly. This is indicated by the lower-right zero block in the system matrix in either formulation. 
Thus, the system does not have a solution if constraints are contradictory. \textcolor{blue}{Aren't $\dot{q} = v$ and $\dot{v} = a$ 
also part of the differential equation? Because $c(v)$ and $f_e$ also depend on $q$!}

In \cite{ascher1995}, the DAE is approached by eliminating the $\lambda$ from the system entirely and constructing an ODE in terms of positions
and velocities. In \cite{tournier2015}, the authors suggest applying implicit integration schemes to the system directly by constructing the 
following Karush-Kuhn-Tucker (KKT) equation system:

\[
\begin{pmatrix}
    \matm{M} & -\matm{J}^T \\
    \matm{J} & 0
\end{pmatrix}
\begin{pmatrix}
    \vecm{v}_{n+1} \\
    \vecm{\mu}_{n+1}
\end{pmatrix}
=
\begin{pmatrix}
    \vecm{p} + h\vecm{f_e} \\
    0
\end{pmatrix}
\]

Here, the external forces $\vecm{f}_{\text{ext}}$ and the constraint gradients $\matm{J}$ are considered constant across the time step 
and $\matm{J}(x_{n+1})$ 
is not approximated using the Taylor expansion like it is in \cite{baraff1998}. If internal forces are taken into account, the upper-left 
matrix $\matm{M}$ is replaced by the matrix $\matm{H}$ from Equation \ref{eq:system-H}.

\textcolor{blue}{Reverse-engineering how the authors arrived at this system is quite enlightening. Start out from the equations of motion 
\cite{ascher1995}
    \begin{align*}
        \dot{\vecm{v}} &= \matm{M}^{-1}(\vecm{f} - \matm{J}^T \vecm{\lambda})
    \end{align*}
    and perform implicit integration:
    \begin{align*}
        \vecm{v}_{n+1} &= \vecm{v}_n + h\matm{M}^{-1}(\vecm{f}_e(\vecm{X}_{n+1}) - \matm{J}^T(\vecm{x}_{n+1}) \vecm{\lambda}(\vecm{x}_{n+1})) \\
        \matm{M}\vecm{v}_{n+1} &= \vecm{p} + h\vecm{f}_e(\vecm{x}_{n+1}) - h\matm{J}^T(\vecm{x}_{n+1})\vecm{\lambda}(\vecm{x}_{n+1}) \\
        \matm{M}\vecm{v}_{n+1} + h \matm{J}^T(\vecm{x}_{n+1})\vecm{\lambda(\vecm{x}_{n+1})} &= \vecm{p} + h\vecm{f}_e(\vecm{x}_{n+1}) \\
        \matm{M}\vecm{v}_{n+1} + \matm{J}^T(\vecm{x}_{n+1}) \vecm{\mu}(\vecm{x}_{n+1}) &= \vecm{p} + h\vecm{f}_e(\vecm{x}_{n+1})
    \end{align*}
}
\textcolor{red}{If we assume that $\vecm{f_e}$ and the constraint gradients $\matm{J}$ are constant across the time step, we arrive at the 
formulation from the paper. For the external forces, which are usually only comprised of gravitational forces, this is not a big 
deal. For the constraint gradients, I am not sure what the ramifications are. In \cite{baraff1998}, the Taylor expansion is performed
which requires the compution of second derivatives over the constraint functions. This is not happening here at all! Is this what 
authors mean when they say that the constraints are effectively linearized during one solve, e.g. second page 
of \cite{mueller2020}? Technically, speaking, even if the Taylor expansion is performed, the constraints are linearized, if I 
understand correctly.}

Note that the system matrix is sparse, which can be exploited by sparse-matrix solvers to solve the system efficiently
\cite{baraff1996}. Alternatively, the Schur complement can be constructed since the mass matrix in the upper left block is invertible.
This leads to a smaller, albeit less sparse system \cite{tournier2015}:

\[
    \matm{J}\matm{M}^{-1}\matm{J}^T \vecm{\mu} = -\matm{J}\matm{M}^{-1}(\vecm{p}) + h \vecm{f}_e
\]

If the constraints are not redundant, $\matm{J}\matm{M}^{-1}\matm{J}^T$ is non-singular and symmetric positive definite \cite{baraff1996}, which are desirable
properties for many solvers. According to \cite{servin2006}, the common approaches for linearizing the constraint forces and stabilizing 
the constraints $\vecm{\phi(x)} = 0$ are notoriously unstable \textcolor{red}{(I need to look this up again. I do not understand what exactly 
this means anymore)}. Additionally, instabilities in the traverse direction of the constraints occur when the tensile force with respect to 
particle masses is large when using hard constraints \cite{tournier2015}.

\subsection{Compliant Constraints}\label{ss:compliant-constraints}
By combining ideas from hard constraints (see Sec.\ \ref{ss:hard-constraints}) and penalty forces (see Sec.\ \ref{ss:penalty-forces}), it is possible to 
formulate the system matrix for hard constraints such that constraints do not have to be enforced exactly. In this approach, called compliant 
constraints, the constraints are combined with Newton's ODE (see Eq.\ \ref{eq:equations-of-motion}) in a way that allows relaxation of constraints in a 
physically meaningful manner \cite{servin2006}. The key insight is that constraints 
of the form $C_j(\vecm{q})$ are the physical limit of strong forces from potentials of the form $\frac{k_j}{2}C_j(\vecm{q})^2$ with 
high stiffness values $k_j$. It can be advantageous to express the system equations in terms of small inverse stiffnesses, also 
called compliances, over working with large stiffness values directly. In particular, setting the compliance to zero often provides
an elegant avenue for modelling infinite stiffness. 

Let $\vecm{C} = [C_i, \ldots, C_r]^T$ be the vector function whose entries consist of the individual constraint 
functions $C_j$. The potential energy for $\vecm{C}$ is then defined as:

\begin{equation}\label{eq:compliant-potential}
    \psi(\vecm{q}) = \frac{1}{2}\vecm{C}(\vecm{q})^T \matm{\alpha}^{-1}\vecm{C}(\vecm{q})
\end{equation}

\noindent where $\matm{\alpha}$ is a positive diagonal matrix given by $\text{diag}(1/k_1, \ldots, 1/k_r)$. The resulting forces are given by 

\begin{equation}\label{eq:compliant-force}
    \vecm{f_c} = \nabla \psi(\vecm{q}) = -\nabla \vecm{C}(\vecm{q})^T\matm{\alpha}^{-1} \vecm{C}(\vecm{q}).
\end{equation}

\noindent Next, artificial variables called Lagrange multipliers $\vecm{\lambda} = -\matm{\alpha}^{-1}\vecm{C}$ are introduced, yielding 

\begin{equation}\label{eq:compliant-force-lambda}
    \vecm{f}_c = \nabla \vecm{C}(\vecm{q})^T\vecm{\lambda}
\end{equation}

\noindent This leads to the following DAE:

\begin{equation}\label{eq:compliant-dae}
\begin{split}
    \dot{\vecm{q}} &= \vecm{v} \\
    \matm{M}\dot{\vecm{v}} &= \vecm{f}_{e} + \nabla \vecm{C}(\vecm{q})^T\vecm{\lambda} \\
    \matm{\alpha}\vecm{\lambda}(\vecm{q}) &= -\vecm{C}(\vecm{q})
\end{split}
\end{equation}

\noindent Note, that if $\matm{\alpha} = \matm{0}$, the formulation from hard constraints is recovered. 

Usually, the DAE is solved by employing some numerical integration scheme (see Sec.\ \ref{ss:numerical-integration}) which eventually requires 
the solution of a linear system of equations. Here, the goal is to arrive at a formulation where both the system matrix and the right 
side of the resulting LSE do not contain references to the large stiffness values
$\matm{\alpha}^{-1}$. This is achieved by treating $\vecm{\lambda} = -\matm{\alpha}^{-1}\vecm{C}(\vecm{q})$ as an unknown, 
pulling it out of the system matrix and hiding all occurrences of the large $\matm{\alpha}^{-1}$ in there. To this end, it is often 
necessary to make simplifying assumptions suitable for the problem at hand. As an example, if the DAE is solved via backward 
differentiation, making the assumption that $\nabla \vecm{C}$ is constant across the time step allows pulling $\vecm{\lambda}$ out of 
the system matrix entirely \cite{tournier2015}. The entries of the resulting system matrix and the corresponding right side
are small, since they do not depend on the large stiffness terms. Backwards differentiation while assuming that $\nabla \vecm{C}$ is 
constant across the time step yields

\[
    -\matm{\alpha\lambda}_{n+1} = \vecm{C}(\vecm{q}_n)\frac{\vecm{\mu}_{n+1}}{h} = -\vecm{C}(\vecm{q}_{n+1}) \approx -\vecm{C}(\vecm{q}_n) - h\nabla 
    \vecm{C}(\vecm{q}_n) \vecm{v}_{n+1},
\]

\noindent leading to the following LSE \cite{tournier2015}

\begin{equation}\label{eq:compliant-tournier}
\begin{pmatrix}
    \matm{M} & -\nabla \vecm{C}(\vecm{q}_n)^T \\
    \nabla \vecm{C}(\vecm{q}_n) & \frac{1}{h^2}\matm{\alpha}
\end{pmatrix}
\begin{pmatrix}
    \vecm{v}_{n+1} \\
    \vecm{\mu}_{n+1}
\end{pmatrix}
=
\begin{pmatrix}
    \vecm{p} + h\vecm{f}_e \\
    - \frac{1}{h}\vecm{C}(\vecm{q}_n)
\end{pmatrix},
\end{equation}

\noindent where $\vecm{\mu} = h\vecm{\lambda}$.

This formulation comes with a couple of advantages. Firstly, relaxing the constraints by keeping a finite but large penalty parameter helps
counteracting numerical problems in the presence of over defined or degenerate constraints. In comparison to the system matrix that arises
for hard constraints \textcolor{red}{(reference)}, the lower-right zero block is replaced by the compliance matrix $\matm{\alpha}$ in 
Equation \ref{eq:compliant-tournier} which prevents the system matrix from becoming singular in the presence of redundant constraints. Thus, the 
compliant 
constraint formulation can be interpreted as a regularization of the hard constraint formulation in (\textcolor{red}{(reference to section)}) 
\cite{tournier2015}. Secondly, setting $\matm{\alpha} = \matm{0}$ allows modelling infinite stiffness since $\matm{\alpha}^{-1}$ does not occur in
Equation \ref{eq:compliant-tournier}. Additionally, Tournier et 
al.\ \cite{tournier2015} claim that, in comparison to penalty forces where high stiffness makes the system matrix in 
\textcolor{red}{(reference equation)} almost singular, decreasing the compliance terms in the lower-right block of the system matrix 
in Equation \ref{eq:compliant-tournier} makes working with high stiffness numerically tractable. However, close inspection of the system matrix of 
Equation \ref{eq:compliant-tournier} shows that it quickly becomes singular for high stiffnesses as well when the lower-right block containing the 
corresponding compliances approaches zero. 

\textcolor{red}{Fabian: Does it? Is it singular if a = 0? Isn't it possibly non-singular depending on grad C? Or am I missing something? 
Maybe explain more "your close inspection".}
