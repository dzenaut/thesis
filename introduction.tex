\chapter{Introduction}\label{ch:introduction}
Visually pleasing simulations of deformable bodies are the cornerstone of many applications of computer graphics like animated movies and video games.
This creates the need for simulation procedures that are capable of handling a variety of non-linear materials including rubber and biological soft 
tissue such as skin, fat and muscles. Realistic results can be achieved through physics-based simulation, where a body's response to deformation is 
calculated in accordance with the equations of motion and its physical material model. However, simulating materials with large material stiffness 
can often prove challenging since large restorative forces in response to even small deformations can cause simulations to become unstable. Similar 
issues arise when it is necessary to exert more direct control over the positions of individual particles. For example, pinning a particle to a fixed 
positions requires the limit of infinitely stiff forces. There, it is often desirable to bypass forces entirely and to manipulate particle positions 
via constraints instead. In real-time applications, the physics update often needs to be performed during a single frame. Since accurate constrained 
physics-based simulation is too expensive to fit into the available time budget, simulation methods that are suitable for real-time applications typically 
trade accuracy and generality for speed. For the reasons outline above, these tradeoffs become particularly apparent when simulating stiff materials 
and when direct control over particle positions is required.

Two popular approaches for real-time physics-based simulation are XPBD \cite{macklin2016} and Projective Dynamics (PD) 
\cite{bouaziz2014}. In XPBD, energy terms are expressed in terms of compliant constraints whose contributing particles are moved closer to physically 
valid positions by independent constraint projections. On the other hand, PD treats numerical integration of the equations of motion as an optimization 
problem that can be solved efficiently by restricting energy potentials to a special structure. PD can be extended to more general energy potentials 
by interpreting it through the lense of the Alternating Direction Method of Multipliers (ADMM) for constrained optimization \cite{overby2017} and 
Quasi-Newton (QN) methods for unconstrained optimization \cite{liu2017}, without sacrificing the efficiency of the original PD algorithm. While the inventors 
of these methods do provide experimental data that highlights each of the solvers' key characteristics, comparisons between simulation methods are usually 
only conducted for simple material models, low material stiffness and a small range of time step sizes. This makes it difficult to evaluate which 
simulation method to pick for specific simulation scenarios, particularly in challenging settings with stiff materials. The goal of this thesis is to 
bridge this gap. The main contributions consist of the following:

\begin{itemize}
    \item A comprehensive theoretical analysis of XPBD and PD-style simulation methods that often -- particularly for XPBD -- goes beyond the 
    analysis provided in the original publications. 
    \item A detailed discussion of the compatibility of XPBD and PD-style simulation methods -- particularly XPBD -- with popular non-linear 
    material models.
    \item Experimental data that suggests that ADMM weights are dependent on the time step size.
    \item An extensive comparison between XPBD and PD-style simulation methods in various settings, including different material models with 
    combinations of a variety of material parameters and time step sizes. In particular, we show that PD-style simulation methods are more 
    robust than XPBD for simulations with competing energy terms and that robust simulations using stiff Neohookean materials can only be achieved 
    using the QN method.
\end{itemize}
