\chapter{Introduction}\label{ch:introduction}
Visually pleasing simulations of deformable bodies are the cornerstone of many applications of computer graphics like animated movies and video games.
This creates the need for simulation procedures that are capable of handling a variety of non-linear materials including rubber and biological soft 
tissue such as skin, fat and muscles. Realistic results can be achieved through physics-based simulation, where a body's response to deformation is 
calculated in accordance with the equations of motion and its physical material model which associates local deformations with energy potentials. 
Additionally, handling common interactions such as collisions, joints or simply attaching different bodies to each other requires the ability to exert 
more direct control over the positions of the simulated bodies. Ideally, this is achieved by introducing hard constraints are guaranteed to be satisfied 
at all time into the simulation.

In real-time applications, the physics update often needs to be performed during a single frame. Since the accurate physics-based simulation is too 
expensive to fit into the time budget for a single frame, physics-based simulation methods that are suitable for real-time applications trade accuracy 
and generality for speed. Two popular approaches for real-time physics-based simulations are XPBD and Projective Dynamics (PD). 
In XPBD, energy terms are expressed in terms of compliant constraints whose contributing particles are moved to closer to physically valid positions 
by independent constraint projections. On the other hand, PD treats numerical integration of the equations of motion as an optimization problem that 
can be solved efficiently by restricting energy potentials to a special structure. PD can be extended to more general energy potentials by reinterpreting 
it through the lense of constrained and unconstrained optimization. 
