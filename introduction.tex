\chapter{Introduction}\label{ch:introduction}
Visually pleasing simulations of deformable bodies are the cornerstone of many applications of computer graphics like animated movies and video games.
This creates the need for simulation procedures that are capable of handling a variety of non-linear materials including rubber and biological soft 
tissue such as skin, fat and muscles. Realistic results can be achieved through physics-based simulation, where a body's response to deformation is 
calculated in accordance with the equations of motion and its physical material model. Additionally, handling common interactions such as collisions, 
joints or simply attaching different bodies to each other requires the ability to exert more direct control over the positions of the simulated bodies. 
Ideally, this is achieved by introducing hard constraints that are guaranteed to be satisfied at all times into the simulation.

In real-time applications, the physics update often needs to be performed during a single frame. Since accurate physics-based simulation is too 
expensive to fit into the available time budget, simulation methods that are suitable for real-time applications trade accuracy 
and generality for speed. Two popular approaches for real-time physics-based simulation are XPBD \cite{macklin2016} and Projective Dynamics (PD) 
\cite{bouaziz2014}. In XPBD, energy terms are expressed in terms of compliant constraints whose contributing particles are moved closer to physically 
valid positions by independent constraint projections. On the other hand, PD treats numerical integration of the equations of motion as an optimization 
problem that can be solved efficiently by restricting energy potentials to a special structure. PD can be extended to more general energy potentials 
by interpreting it through the lense of the Alternating Direction Method of Multipliers (ADMM) for constrained optimization \cite{overby2017} and 
Quasi-Newton methods for unconstrained optimization \cite{liu2017}, without sacrificing the efficiency of the original PD algorithm. 

While the inventors of the methods above do provide experimental data that highlights each of the solvers' key characteristics, comprehensive 
comparisons between different approaches are difficult to come by. In particular, comparisons between simulation methods are usually only conducted 
for simple material models, low stiffness values and a small range of time step sizes. This makes it challenging to evaluate which simulation method 
is most suitable for a specific simulation scenario. The goal of this thesis is to bridge this gap. The main contributions consist of the following:

\begin{itemize}
    \item A comprehensive theoretical analysis of each of the simulation methods that often -- particularly for XPBD -- goes beyond the analysis 
        provided in the original publications. 
    \item A detailed discussion of the compatibilty of the simulation methods -- particularly XPBD -- with popular non-linear material models.
    \item Experimental data that suggests that ADMM weights are dependent on the time step size.
    \item An extensive comparison between the simulation methods in various settings, including different material models with combinations of 
        a variety of material parameters and time step sizes.
\end{itemize}
