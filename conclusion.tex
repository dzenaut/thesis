\chapter{Conclusion}\label{ch:conclusion}
XPBD and PD-style methods, including the original PD method and its extensions based on the Alternating Direction Method of 
Multipliers (ADMM) and Quasi-Newton methods (QN), are popular choices for real-time physics-based simulation. Despite their 
popularity, there is a lack of experimental data demonstrating the suitability of each method for different simulation 
scenarios. This thesis addresses this gap by comparing these methods in various settings with different non-linear material 
models, spanning a wide range of material stiffness values and time step sizes. Our analysis reveals that ADMM and, to a 
lesser extent, XPBD are not suited for simulating complex non-linear materials with significant stiffness, as exemplified 
by the Neohookean material model. Due to the incompatibility of Neohookean energy terms with PD energy potentials, the QN 
method emerges as the most reliable option for robust simulation of stiff non-linear materials. Moreover, we observe that 
XPBD can often introduce noticeable geometric artifacts when handling incompatible stiff constraints. Despite its favorable 
convergence properties, achieving precise control over individual particle positions remains challenging with the QN solver.

While our study provides extensive experimental data comparing XPBD and PD-style solvers, it is not exhaustive. Notably, 
we lack experiments evaluating the ability of solvers to precisely control particle positions. In scenarios where strict 
enforcement of particle positions is critical, XPBD and ADMM may offer advantages over QN, despite their generally poorer 
convergence properties in our experiments with soft position constraints. Additionally, our experiments with the Neohookean 
model prioritize modelling large material stiffness over strict volume preservation, which differs from the focus of 
Macklin and Müller \cite{macklin2021}. They propose that XPBD performs well in scenarios where materials are nearly 
incompressible. Future work could expand our this analysis to include a broader range of material models.

Lastly, Chen et al.\ \cite{chen2024} recently presented a method called Vertex Block Descent that is similar to XPBD, but 
performs updates on individual particles instead of groups of particles contributing to a single constraint. This approach
supports massive parallelization, enabling the simulation of complex geometries at competitive frame rates. It would 
be interesting to extend our analysis to this method as well.

