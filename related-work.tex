\chapter{Related Work}\label{ch:related-work}
In their presentations of XPBD \cite{macklin2016} and PD-style solvers, including the original PD solver \cite{bouaziz2014} and its 
extensions using ADMM \cite{overby2017} and Quasi-Newton (QN) methods \cite{liu2017}, the authors offer experimental data showcasing 
the key properties of each method. A brief summary is provided below.

\paragraph{XPBD.}
Macklin et al.\ \cite{macklin2016} demonstrate that XPBD overcomes the problem of time step dependent constraint stiffness of its 
predecessor PBD \cite{mueller2006} by example of cloth simulations using geometric constraints. Additionally, the authors compare 
XPBD to Newton's method, demonstrating that XPBD achieves visually indistinguishable results from Newton's method in simple one- 
and two-dimensional scenarios. Macklin and Müller \cite{macklin2021} later propose a method for modelling Neohookean materials using 
XPBD. Here, the stability of the resulting solver is demonstrated for varying degrees of material incompressibility, controlled by 
the Poisson ratio. However, the material stiffness (controlled by Youngs Modulus) is kept low and fixed.

\paragraph{PD.}
Of the methods discussed in this thesis, PD dates back the furthest. Bouaziz et al. \cite{bouaziz2014} primarily compare their solver 
to XPBD's predecessor, PBD. However, since PBD can be easily extended to XPBD without any drawbacks, XPBD has largely replaced PBD 
in recent years. Consequently, the experimental data provided by Bouaziz et al.\ \cite{bouaziz2014} offers limited insights into the differences 
between PD and XPBD.

\paragraph{ADMM.}
Overby et al.\ \cite{overby2017} compare ADMM to PD and an L-BFGS method, which differs from the one proposed by Liu et al.\ 
\cite{liu2017}. They evaluate the convergence rates of these methods by examining how quickly they minimize the objective function of 
the variational form of implicit Euler integration. However, their experiments are limited to a simple non-linear material model 
with fixed material stiffness.

\paragraph{QN.}
During their presentation of a Quasi-Newton method for the variational form of implicit Euler integration based on PD, Liu et 
al.\ \cite{liu2017} evaluate solver convergence using the same methodology as Overby et al.\ \cite{overby2017}. While their 
experimental data is still limited to fixed material stiffness, they use the more complex non-linear Neohookean material model. 
Liu et al.\ \cite{liu2017} emphasize the differences between their L-BFGS method and other methods for unconstrained 
optimization, comparing it to L-BFGS methods with different initial Hessian approximations, conjugate gradient solvers, and Newton's method.

\paragraph{}
\noindent Overall, the summary above highlights the scarcity of experimental data comparing XPBD and PD-style methods. Additionally, 
comparisons between these simulation methods often focus on linear or simple non-linear material models with fixed material stiffness. 
As such, the experimental data only provides a limited view of the simulation methods' characteristics in specific settings. Given the 
current data, it is challenging to assess each method's suitability for more complex scenarios involving non-linear material models and 
significant material stiffness.
